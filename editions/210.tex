\documentclass{article}

\usepackage[table]{xcolor}
\usepackage[utf8]{inputenc}
\usepackage[parfill]{parskip}
\usepackage{tabulary}
\PassOptionsToPackage{hyphens}{url}
\usepackage{hyperref}    
\usepackage[capitalize]{cleveref}
\begin{document}
\title{SIGLOG Monthly 210}\date{February 01, 2021}\maketitle


\href{https://lics.siglog.org/newsletters/}{Past Issues}
 - 
\href{https://lics.siglog.org/newsletters/inst.html}{How to submit an announcement}
\section{Table of Content}\begin{itemize}\item DEADLINES (\cref{deadlines}) 
 
\item SIGLOG MATTERS 
 
\begin{itemize}\item The 2021 Alonzo Church Award for Outstanding Contributions to Logic and Computation (\cref{The2021AlonzoChurchAwardforOutstandingContributionstoLogicandComputation})
\end{itemize} 
\item COVID UPDATES 
 
\begin{itemize}\item ICALP 2021 (\cref{ICALP2021})
\end{itemize} 
\item CALLS 
 
\begin{itemize}\item TARK 2021 (CALL FOR SUBMISSIONS) (\cref{TARK2021})
\item NALOMA'21 (CALL FOR PAPERS) (\cref{NALOMA21})
\item Diagrams 2021 (CALL FOR PAPERS) (\cref{Diagrams2021})
\item E. W. Beth Outstanding Dissertation Prize 2021 (CALL FOR NOMINATIONS) (\cref{EWBethOutstandingDissertationPrize2021})
\item MFCS 2021 (CALL FOR PAPERS) (\cref{MFCS2021})
\end{itemize} 
\item JOB ANNOUNCEMENTS 
 
\begin{itemize}\item Associate Professor/Professor of Programming Languages (\cref{AssociateProfessorProfessorofProgrammingLanguages})
\end{itemize} 
\end{itemize}\section{Deadlines}\label{deadlines}\rowcolors{1}{white}{gray!25}\begin{tabulary}{\linewidth}{LL}LAMAS\&SR:  & Feb 10, 2021 (Paper) \\
ICALP 2021:  & Feb 12, 2021 (Paper) \\
FSDC 2021:  & Feb 12, 2021 (Abstracts), Feb 15, 2021 (Full papers) \\
QPL 2021:  & Feb 12, 2021 (Paper deadline) \\
CADE-28:  & Feb 15, 2021 (Abstracts), Feb 22, 2021 (Full papers) \\
J. LOG. COMPUT (Social Networks):  & Feb 15, 2021 (Extended (again)) \\
SPIN 2021:  & Mar 1, 2021 (Paper) \\
Alonzo Church Award:  & Mar 1, 2021 (Deadline for Nominations) \\
TARK 2021:  & Mar 15, 2021 (Abstract), Mar 20, 2021 (Extended abstract) \\
NALOMA'21:  & Mar 26, 2021 (Papers \& extended abstracts) \\
Diagrams 2021:  & Apr 01, 2021 (Titles+short abstracts), Apr 08, 2021 (Long and Short Papers), Apr 15, 2021 (Abstracts and Posters) \\
FORMATS 2021:  & Apr 06, 2021 (Abstract), Apr 13, 2021 (Paper) \\
E. W. Beth Outstanding Dissertation Prize 2021:  & Apr 15, 2021 (Deadline for nominations) \\
MFCS 2021:  & Apr 30, 2021 (Abstract), May 03, 2021 (Paper) \\
\end{tabulary}
\section{The 2021 Alonzo Church Award for Outstanding Contributions to Logic and Computation}\label{The2021AlonzoChurchAwardforOutstandingContributionstoLogicandComputation}CALL FOR NOMINATIONS 

\begin{itemize}\item  INTRODUCTION  
 
  An annual award, called the Alonzo Church Award for Outstanding Contributions to Logic and Computation, was established in 2015 by the ACM Special Interest Group for Logic and Computation (SIGLOG), the European Association for Theoretical Computer Science (EATCS), the European Association for Computer Science Logic (EACSL), and the Kurt Gödel Society (KGS). The award is for an outstanding contribution represented by a paper or by a small group of papers published within the past 25 years. This time span allows the lasting impact and depth of the contribution to have been established.  The award can be given to an individual, or to a group of individuals who have collaborated on the research. For the rules governing this award, see \href{https://siglog.org/alonzo-church-award/}{https://siglog.org/alonzo-church-award/}, \href{https://www.eatcs.org/index.php/church-award/}{https://www.eatcs.org/index.php/church-award/}, and \href{https://eacsl.org/}{https://eacsl.org/}.  
 
  The 2020 Alonzo Church Award was given jointly to Ronald Fagin, Phokion G. Kolaitis, Renée J. Miller, Lucian Popa, and Wang Chiew Tan for their ground-breaking work on laying the logical foundations for data exchange. Lists containing this and all previous winners can be found through the links above.  
 
\item  ELIGIBILITY AND NOMINATIONS 
 
  The contribution must have appeared in a paper or papers published within the past 25 years. Thus, for the 2021 award, the cut-off date is January 1, 1996. When a paper has appeared in a conference and then in a journal, the date of the journal publication will determine the cut-off date. In addition, the contribution must not yet have received recognition via a major award, such as the Turing Award, the Kanellakis Award, or the Gödel Prize. (The nominee(s) may have received such awards for other contributions.) While the contribution can consist of conference or journal papers, journal papers will be given a preference. 
 
  Nominations for the 2021 award are now being solicited. The nominating letter must summarize the contribution and make the case that it is fundamental and outstanding. The nominating letter can have multiple co-signers. Self-nominations are excluded. Nominations must include: a proposed citation (up to 25 words); a succinct (100-250 words) description of the contribution; and a detailed statement (not exceeding four pages) to justify the nomination. Nominations may also be accompanied by supporting letters and other evidence of worthiness. 
 
  Nominations should be submitted to javier.esparza@in.tum.de by March 1, 2021. 
 
\item  PRESENTATION OF THE AWARD 
 
  The 2021 award will be presented at the ACM SIGLOG/IEEE Symposium on Logic in Computer Science, LICS 2021, which is scheduled to take place in Rome in June/July 2021. The award will be accompanied by an invited lecture by the award winner, or by one of the award winners. The awardee(s) will receive a certificate and a cash prize of USD 2,000. If there are multiple awardees, this amount will be shared. 
 
\item  AWARD COMMITTEE 
 
  The 2021 Alonzo Church Award Committee consists of the following five members: Mariangiola Dezani, Thomas Eiter, Javier Esparza (chair), Radha Jagadeesan, and Igor Walukiewicz. 
 
\end{itemize}\section{ICALP 2021:}\label{ICALP2021}COVID UPDATE 

\begin{itemize}\item  ICALP 2021 will be online only. 
 
\end{itemize}\section{TARK 2021: The Eighteenth Conference on Theoretical Aspects of Rationality and Knowledge}\label{TARK2021}  June 25-27, 2021 at Tsinghua University, Beijing, China.\\ 
  Format: At this moment it is still unknown in which format the conference will take place: in-person, online, or (most likely) a combination of the two.\\ 
  \href{http://tsinghualogic.net/JRC/?page_id=2034}{http://tsinghualogic.net/JRC/?page\_id=2034}\\ 
CALL FOR SUBMISSIONS 

\begin{itemize}\item  The mission of the TARK conferences is to bring together researchers from a wide variety of fields, including Computer Science, Artificial Intelligence, Game Theory, Decision Theory, Philosophy, Logic, Linguistics, and Cognitive Science, in order to further our understanding of interdisciplinary issues involving reasoning about rationality and knowledge. Previous conferences have been held bi-annually around the world. The information of all previous TARK conferences can be accessed at \href{http://www.tark.org}{http://www.tark.org} 
 
\item  TOPICS OF INTEREST 
 
  Topics include, but are not limited to, semantic models for knowledge, belief, awareness and uncertainty, bounded rationality and resource-bounded reasoning, commonsense epistemic reasoning, epistemic logic, epistemic game theory, knowledge and action, applications of reasoning about knowledge and other mental states, belief revision, and foundations of multi-agent systems. 
 
\item  CONTENT 
 
  Strong preference will be given to papers whose topic is of interest to an interdisciplinary audience, and papers should be accessible to such an audience. Papers will be held to the usual high standards of research publications. In particular, they should: 
 
\begin{itemize}\item  contain enough information to enable the program committee to identify the main contribution of the work;
\item  explain the significance of the work -- its novelty and its practical or theoretical implications; and
\item  include comparisons with and references to relevant literature.
\end{itemize} 
\item  SUBMISSIONS 
 
  \href{https://easychair.org/conferences/?conf=tark2021}{https://easychair.org/conferences/?conf=tark2021} 
 
  One author of each accepted paper will be expected to present the paper at the conference. Abstracts should be no longer than 10 pages. Optional technical details such as proofs may be included in an appendix.  Note that the 10 page limit is to ensure that the reviewers can read and express an opinion on the submission within short time, though the submission format compresses the paper considerably. Please ensure that the main text for the reviewers stays within this limit. 
 
  To format your paper please use: LaTeX2e - Tighter Alternate style from \href{http://www.acm.org/sigs/publications/proceedings-templates}{http://www.acm.org/sigs/publications/proceedings-templates} without anonymisation (not double-blind). 
 
\item  PROCEEDINGS: 
 
  There will be a proceedings for TARK 2021, at EPTCS (Electronic Proceedings in Theoretical Computer Science). The proceedings of previous TARK conferences can be accessed at \href{http://www.tark.org/}{http://www.tark.org/} . The proceedings of TARK 2021 will also be open access and available online. 
 
\item  IMPORTANT DATES: 
 
\rowcolors{1}{white}{gray!25}\begin{tabulary}{\linewidth}{LL}Abstract submission:  & Mar 15, 2021 \\
Extended abstract submission:  & Mar 20, 2021 \\
Notification of acceptance:  & Apr 20, 2021 \\
Early registration:  & May 01, 2021 \\
Camera ready version for proceedings:  & May 10, 2021 \\
Registration:  & Jun 15, 2021 \\
\end{tabulary}
 
\end{itemize}\section{NALOMA'21: Second NAtural LOgic meets MAchine Learning }\label{NALOMA21}  Online, Netherlands\\ 
  \href{https://typo.uni-konstanz.de/naloma21/}{https://typo.uni-konstanz.de/naloma21/}\\ 
CALL FOR PAPERS 

\begin{itemize}\item   After the successful completion of NALOMA'20 (NAtural LOgic Meets MAchine Learning), NALOMA’21 seeks to continue the series and attract exciting contributions. The workshop aims to bridge the gap between ML/DL and symbolic/logic-based approaches to NLI, and it is perhaps the only workshop organized to do so. It will take place from June 14-June 18, 2021, during the International Conference on Computational Semantics (IWCS 2021) organized by the University of Groningen but taking place fully online due to the pandemic. 
 
  NALOMA'21 is set out to address two main issues of the NLI community. First, the approaches and systems currently used to address NLI are too one-dimensional, and no fruitful dialog between them is promoted. One strand of research focuses on training large DL models that can achieve what has been identified as ``human performance''. With the world-knowledge that is encapsulated in such models and their robust nature, such approaches can deal with diverse and large data in an efficient way. However, it has been repeatedly shown that such models lack generalization power and are far from solving NLI. When presented with differently biased data or with complex inferences containing hard linguistic phenomena, they struggle to reach the baseline. Explicitly detecting and solving these weaknesses is only partly possible, e.g., through appropriate datasets, because such models act like black-boxes with low explainability. Another strand of research targets more traditional approaches to reasoning, employing some kind of logic or semantic formalism. Such approaches excel in precision, especially of complex inferences with hard linguistic phenomena, e.g., negation, quantifiers, modals, etc. However, they suffer from inadequate world-knowledge and lower robustness, making it hard for them to compete with the state-of-the-art models. Overall, current methods to NLI are too one-dimensional: they are either purely DL or purely symbolic but do not attempt to combine the two worlds. A second issue concerns datasets. Existing NLI datasets are either complex enough but too small to be used for proper learning, e.g., the FraCas or the RTE datasets, or large enough but too easy to be claimed to represent human inference, e.g., SICK, SNLI, MNLI, etc. 
 
\item  SUBMISSIONS 
 
  The workshop invites submissions on any (theoretical or computational) topic concerning NLI, including but not limited to: 
 
\begin{itemize}\item  hybrid NLI systems integrating symbolic/logic-based methods with ML/DL approaches (particularly, approaches combining Natural Logic with ML/DL)
\item  explainable models of NLI
\item  opening the ``black box'' of NLI models
\item  probabilistic semantics for NLI
\item  downstream applications of NLI
\item  creation, evaluation, and criticism of NLI datasets,
\item  theoretical notions and refinement of the NLI task to address inherent disagreements
\item  comparison and contrast between human-level and machine-level work in NLI
\item  using symbolic/logic-based methods for data cleaning and augmentation
\item  NLI for other languages than English
\end{itemize} 
  We invite two types of submission: 
 
\begin{itemize}\item  Archival (long or short) papers should report on complete, original and unpublished research. Accepted papers will be published in the workshop proceedings and appear in the ACL anthology. 
\item  Extended abstracts may report on work in progress or work that was recently published/accepted at a different venue. Extended abstracts will not be included in the workshop proceedings. Thus, the unpublished work will retain the status and can be submitted to another venue. This webpage will link to the accepted extended abstracts.
\end{itemize} 
  Both accepted papers and extended abstracts are expected to be presented at the workshop. Extended abstracts will be presented as talks or posters at the discretion of the program committee. 
 
  Authors must submit non-anonymized extended abstracts or papers by March 26. Both extended abstracts and papers must be formatted according to the IWCS guidelines (available soon). The extended abstracts should not contain an abstract section and may consist of up to 2 pages of content, plus unlimited references. Short and long papers may consist of up to 4 and 8 pages of content, respectively, plus unlimited references. Camera-ready versions of papers will be given one additional page of content so that reviewers’ comments can be taken into account. 
 
  Both extended abstracts and follow-up papers should be submitted via SoftConf (link will be available soon on the conference website). 
 
\item  DATES 
 
\rowcolors{1}{white}{gray!25}\begin{tabulary}{\linewidth}{LL}Papers \& extended abstracts submission:  & Mar 26, 2021 \\
Workshop Dates:  & Jun 14-18, 2021 \\
\end{tabulary}
 
\item  For further information, including the program committee and invited speakers, please see \href{https://typo.uni-konstanz.de/naloma21/}{https://typo.uni-konstanz.de/naloma21/} 
 
\end{itemize}\section{Diagrams 2021: 12th International Conference on the Theory and Application of Diagrams}\label{Diagrams2021}  September 28 – 30, 2021\\ 
  Virtual Event\\ 
  \href{http://www.diagrams-conference.org/2021}{http://www.diagrams-conference.org/2021}\\ 
CALL FOR PAPERS 

\begin{itemize}\item  Diagrams is the only conference series that provides a united forum for all areas that are concerned with the study of diagrams and has a multidisciplinary emphasis. 
 
\begin{itemize}\item  FREE REGISTRATION
\item  Proceedings published by Springer
\item  Three Tracks: Main, Philosophy, and Psychology and Education.
\item  Graduate Symposium
\item  Best Paper and Best Student Paper awards
\item  Submission dates in April 2021.
\end{itemize} 
\item  TOPICS 
 
\begin{itemize}\item  applications of diagrams,
\item  computational models of reasoning with, and interpretation of, diagrams,
\item  design of diagrammatic notations,
\item  diagram understanding by humans or machines,
\item  diagram aesthetics and layout, evaluation of diagrammatic notations,
\item  graphical communication and literacy,
\item  heterogeneous notations involving diagrams,
\item  history of diagrammatic notations,
\item  information visualization using diagrams,
\item  nature of diagrams and diagramming,
\item  novel technologies for diagram use,
\item  reasoning with diagrams,
\item  semiotics of diagrams,
\item  software to support the use of diagrams, and
\item  usability and human-computer interaction issues concerning diagrams.
\end{itemize} 
\item  In addition to the main track, Diagrams 2021 will have two further tracks: Philosophy, and Psychology and Education; for their topics of interest, see the website. If the main research contribution of your submission is considered to fit either of the other tracks then you are strongly encouraged to submit to the respective special track, each of which has a dedicated program committee. 
 
\item  Authors of accepted submissions will be expected to be in attendance at the virtual conference to present their research and respond to questions presented by delegates. 
 
\item  Submission Categories 
 
  The conference will include presentations of refereed Papers, Abstracts, and Posters, alongside a graduate symposium. We invite submissions for peer review that focus on any aspect of diagrams research, as follows: 
 
\begin{itemize}\item  Long Papers (16 pages),
\item  Abstracts (3 pages),
\item  Short Papers (8 pages),
\item  Posters (4 pages – this is both a maximum and minimum requirement).
\end{itemize} 
\item  Call for Graduate Symposium: \href{http://www.diagrams-conference.org/2021/index.php/calls/graduate-symposium/}{http://www.diagrams-conference.org/2021/index.php/calls/graduate-symposium/} 
 
\item  Call for tutorials: \href{http://www.diagrams-conference.org/2021/index.php/calls/tutorials/}{http://www.diagrams-conference.org/2021/index.php/calls/tutorials/} 
 
\item  DATES 
 
\rowcolors{1}{white}{gray!25}\begin{tabulary}{\linewidth}{LL}Titles+short abstracts:  & Apr 01, 2021 \\
Long and Short Papers:  & Apr 08, 2021 \\
Abstracts and Posters:  & Apr 15, 2021 \\
Rebuttal phase:  & Jun 3 – 7, 2021 \\
Notification:  & Jun 16, 2021 \\
Camera ready deadline (FIRM):  & Jul 05, 2021 \\
\end{tabulary}
 
\end{itemize}\section{E. W. Beth Outstanding Dissertation Prize 2021}\label{EWBethOutstandingDissertationPrize2021}CALL FOR NOMINATIONS 

\begin{itemize}\item  Since 2002, the Association for Logic, Language, and Information (FoLLI) has been awarding the annual E.W. Beth Dissertation Prize to outstanding Ph.D. dissertations in Logic, Language, and Information (\href{http://www.folli.info/?page_id=74}{http://www.folli.info/?page\_id=74}), with financial support of the E.W. Beth Foundation (\href{https://www.knaw.nl/en/awards/funds/evert-willem-beth-stichting/evert-willem-beth-foundation}{https://www.knaw.nl/en/awards/funds/evert-willem-beth-stichting/evert-willem-beth-foundation}). Nominations are now invited for the best dissertation in these areas resulting in a Ph.D. degree awarded in 2020. 
 
Deadline for nominations: Apr 15, 2021 
 
\item  Qualifications: 
 
\begin{itemize}\item  A Ph.D. dissertation on a topic concerning Logic, Language, or Information is eligible for the Beth Dissertation Prize 2021, if the degree was awarded  between January 1st and December 31st, 2020.
\item  There are no restrictions on the nationality, ethnicity, age, gender or employment status of the author of the nominated dissertation, nor on the university, academic department or scientific institution formally conferring the Ph.D. degree, nor on the language in which the dissertation has originally been written.
\item   In accordance with the aim of the Beth Foundation to continue and extend the work of the Dutch logician Evert Willem Beth, nominations are invited of excellent dissertations on topics in the broad remit of ESSLLI, including current topics in philosophical and mathematical logic, computer science logic, philosophy of science, philosophy of language, history of logic, history of the philosophy of science and scientific philosophy in general, as well as the current theoretical and foundational developments in information and computation, language, and cognition. Dissertations with results more broadly impacting various research areas in their interdisciplinary investigations are especially solicited.
\item   If a nominated dissertation has originally been written in a language other than English, its dossier should still contain the required 10 page English abstract, see below. If the committee decides that a nominated dissertation in a language other than English requires translation to English for proper evaluation, the committee can transfer its nomination to the competition in 2022. The English translation must in such cases be submitted before the deadline of the call for nominations in 2022. The committee may recommend the Beth Foundation to consider supporting such nominated dissertations for English translation, upon request by the author of the dissertation.
\end{itemize} 
\item  The prize consists of: 
 
\begin{itemize}\item  a certificate
\item  a donation of 3000 euros, provided by the E.W. Beth Foundation
\item  an invitation to submit the dissertation, possibly after revision, for publication in FoLLI Publications on Logic, Language and Information (Springer).
\end{itemize} 
\item  Only digital submissions are accepted, without exception. Hard copy submissions are not allowed. The following documents are to be submitted in the nomination dossier: 
 
\begin{itemize}\item  The original dissertation in pdf format (ps/doc/rtf etc. not acceptable).
\item  A ten-page English abstract of the dissertation, presenting the main results of each chapter.
\item  A letter of nomination from the dissertation supervisor, which concisely describes the scope and significance of the dissertation, stating when the degree was officially awarded and the members of the Ph.D. committee. Nominations should contain the address, phone and email details of the nominator.
\item  Two additional letters of support, including at least one from a referee not affiliated with the academic institution that awarded the Ph.D. degree, nor otherwise related to the nominee (e.g. former teachers, supervisors, co-authors, publishers or relatives) or the dissertation.
\item  Self-nominations are not possible.
\end{itemize} 
 All pdf documents must be submitted electronically, as one zip file, via EasyChair by following the link \href{https://easychair.org/conferences/?conf=bodp2021}{https://easychair.org/conferences/?conf=bodp2021}. In case of any problems or questions please contact the chair of the committee Mehrnoosh Sadrzadeh (m.sadrzadeh@ucl.ac.uk). 
 
\item  The prize will be awarded by the chair of the FoLLI board at a ceremony during the 32nd ESSLLI summer school in Utrecht, August 2-13, 2021. 
 
\item  FoLLI is committed to diversity and inclusion and we welcomes dissertations from all under-represented groups 
 
\end{itemize}\section{MFCS 2021: The 46th International Symposium on Mathematical Foundations of Computer Science}\label{MFCS2021}  August 23-27, 2021, Tallinn, Estonia\\ 
  \href{https://compose.ioc.ee/mfcs/}{https://compose.ioc.ee/mfcs/}\\ 
CALL FOR PAPERS 

\begin{itemize}\item  The MFCS conference series has been organised since 1972. Traditionally, the conference moved between the Czech Republic, Slovakia, and Poland, while since 2013, the conference travels around Europe. In 2021, it will come to Tallinn, Estonia. MFCS is a high quality venue for original research in all branches of theoretical computer science. 
 
\item  SUBMISSION GUIDELINES 
 
  Papers should be submitted electronically through EasyChair at \href{https://easychair.org/conferences/?conf=mfcs2021}{https://easychair.org/conferences/?conf=mfcs2021} 
 
  Submissions must be formatted using the LIPIcs style with length not exceeding 12 pages (excluding references and an optional appendix - to be consulted at the discretion of the program committee).  
 
  No prior publication or simultaneous submission to other conferences or journals are allowed (except preprint repositories such as arXiv or workshops without formal published proceedings). 
 
  MFCS 2021 proceedings will be published in LIPIcs (Leibniz International Proceedings in Informatics) under an open source license (as in previous years). 
 
\item  LIST OF TOPICS 
 
  We encourage submission of original research papers in all areas of theoretical computer science, including (but not limited to) the following: 
 
\begin{itemize}\item  algebraic and co-algebraic methods in computer science
\item  algorithms and data structures
\item  automata and formal languages
\item  bioinformatics 
\item  combinatorics on words, trees, and other structures 
\item  computational complexity (structural and model related) 
\item  computational geometry 
\item  computer aided verification 
\item  computerassisted reasoning 
\item  concurrency theory 
\item  cryptography and security 
\item  cyber physical systems, databases and knowledgebased systems 
\item  formal specifications and program development 
\item  foundations of computing 
\item  logics in computer science 
\item  mobile computing 
\item  models of computation 
\item  networks 
\item  parallel and distributed computing 
\item  quantum computing 
\item  semantics and verification of programs 
\item  theoretical issues in artificial intelligence and machine learning 
\item  types in computer science
\end{itemize} 
\item  IMPORTANT DATES (AoE) 
 
\rowcolors{1}{white}{gray!25}\begin{tabulary}{\linewidth}{LL}Abstract submission:  & Apr 30, 2021 \\
Paper submission:  & May 03, 2021 \\
Notification:  & Jun 21, 2021 \\
Conference:  & Aug 23-27, 2021 \\
\end{tabulary}
 
\end{itemize}\section{Associate Professor/Professor of Programming Languages}\label{AssociateProfessorProfessorofProgrammingLanguages}  University of Oxford and University College, Oxford\\ 
  Full details: \href{http://www.cs.ox.ac.uk/news/1870-full.html}{http://www.cs.ox.ac.uk/news/1870-full.html}\\ 
JOB ANNOUNCEMENT 

\begin{itemize}\item  The Department of Computer Science at the University of Oxford, together with  
 
  University College, Oxford, seek to appoint an associate professor/professor in programming languages, with a tutorial fellowship at University College. The post is based in the Department of Computer Science and University College, to start before October 2021. You will also be appointed as a Fellow and Tutor in Computer Science at University College. Tutors are responsible for the organisation and teaching of their subject within the College. You will be a member of both the University and the College community, part of a lively and intellectually stimulating research community with access to the excellent research facilities which Oxford offers. You will play a role in the running of the College as a member of the Governing Body and a trustee of the College as a charity, and have opportunities to interact with academics in other disciplines as part of Oxford's unique collegiate system. The Department is a vibrant and growing academic department, which has a research profile across the entire spectrum of contemporary computing. You will be expected to engage in independent and original research in the area of programming languages, securing funding and engaging in the management of research projects, and disseminate research of the highest international standard through publications, conferences and seminars. You will also contribute to teaching on the Department's highly successful undergraduate and graduate programmes. 
 
\item  REQUIREMENTS 
 
  You will hold a doctoral degree in Computer Science (or cognate discipline), have the ability to teach across a range of Computer Science subjects, and will also have a proven research record of high quality at international level in the area of Programming Languages, and experience of research collaborations at both national and international level.   
 
\item  SALARY 
 
  From: £48,114 p.a. (plus benefits including additional pensionable benefits including  college housing allowance of 11,246 p.a. at current rates, or access to joint equity scheme,  and private health insurance scheme). (An allowance of £2,804 p.a. would be payable upon award of Full Professor title).  
 
\item  DATES 
 
\begin{itemize}\item  Closing date for applications: 12 noon on 15th February 2021
\item  Interviews are expected to be held on 22nd March 2021
\end{itemize} 
\end{itemize}


To the \href{http://siglog.org/}{SIGLOG} or \href{https://lics.siglog.org}{LICS} website\end{document}